%!TEX program = xelatex
\documentclass[letterpaper,12pt]{exam}
\usepackage{csc274videonotes}

\newcommand{\unit}{Unit 08}
\pagestyle{headandfoot}
\firstpageheader{CSC 274 \semester\ \  \unit}{}{Name: $\rule{6cm}{0.15mm}$}
\runningheader{CSC 274 \semester}{\unit}{Page \thepage\ of \numpages}
\firstpagefooter{}{}{}
\runningfooter{}{}{}

\begin{document}

\begin {questions}

\section*{\unit\_010 --- Multiprocessing} % Unnumbered section

\begin{samepage}
	\question What is the CPU?
	\vspace{5mm}
\end{samepage}

\begin{samepage}
\question Why is it often handy to have two windows open while writing scripts? 
\vspace{5mm}
\end{samepage}

\begin{samepage}
\question What was the \code{sed} command used for in the script? (That is, what did it accomplish?)
\vspace{5mm}
\end{samepage}


\begin{samepage}
\question  What was the \code{wc} command used for in the script?
\vspace{5mm}
\end{samepage}

\section*{\unit\_015 --- More Hardware Information}

\begin{samepage}
\question What does each of the following do? 
\begin{itemize}
\item \code{lshw}
\vspace{5mm}
\item \code{lscpu}
\vspace{5mm}
\item \code{lsusb}
\vspace{5mm}
\item \code{lsblk}
\vspace{5mm}
\item \code{lsdev}
\vspace{5mm}
\end{itemize}
\end{samepage}

\section*{\unit\_020 --- Memory}

\begin{samepage}
\question What is RAM? 
\vspace{5mm}
\end{samepage}

\question What is SWAP on Unix systems?
\begin{samepage}
\question How many bytes in a Kilobyte?
\question How many Kilobytes in a Megabyte?
\question How many Megabytes in a Gigabyte?
\question How many Gigabytes in a Terabyte?
\question How many Terabytes in a Petabyte?
\end{samepage}

\question What is $2^{10}$?  Are you sick of writing it yet? (don't answer the second question!)
\vspace{20mm}
\begin {samepage}

\noindent \textbf{Warstory:}  CPU speed is measured in Hertz.  1 hertz is 1 beat per second.  CPU speeds used to be measured in MHz (Megahertz).  The MHz kept creeping up to 800 MHz, 900MHz, and eventually hit 1000MHz.  This gave advertisers a problem.  Consumers had been trained to think more MHz was good.  So consumer-oriented advertisements would advertise 1000MHz.  But if the computer was being marketed to techies, the same system would be marketed as 1 GHz.  Eventually consumers figured out that they really wanted to have one of those "Gigahertz things" on their computer.  
\question One day a student brought in a new advertisement for an old computer.  It was a very low price on a computer with one of those "Gigahertz things."  The add said that the CPU was .8 GHz.   What was the speed of this computer in MHz?
\vspace{15mm}

\end{samepage}
\begin{samepage}
\question What is cache?  
\vspace{5mm}
\end{samepage}

\question What does the \code{free} command do?

\section*{\unit\_030 --- Processes}

\question What is a process?
\vspace{10mm}

\question Would this regular expression recognize a process id?  \texttt{\textbf{\textquotesingle\^{}[0-9]\$\textquotesingle}}?  Explain your answer.
\vspace{10mm}

\question What command shows processes associated with the current terminal session?
\vspace{5mm}

\question What command shows all processes being run by the user, even if they are not associated with this terminal?
\vspace{5mm}

\noindent Comment:  In the video I didn't really explain what the \code{nice} command does.  Let's say you have a big program that might take several minutes or hours to run.  You may plan to do something like take a coffee break and come back later to see the results.  In that case you might use nice to put a low priority on your process.  That way other users of the system would not be slowed down.

\section*{X11 server}

\begin{samepage}
\question What is X-11 forwarding? 
\vspace{5mm}
\end{samepage}
\noindent Comment:  The heading "Who needs X11?" should be "Who needs to install an X11 server"

\noindent Comment The Windows 10 X11 server thing does not seem to be working.  The current news is that it will be working in Windows 11.  We'll see.







\end{questions}
\noindent 

\begin{figure}[b]\label{end}
	\center
	\includegraphics[width=1in]{tux}
	%{\center{By Maxo based on the work File:Tux-G2.png}}
\end{figure}
\end{document}