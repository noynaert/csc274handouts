%!TEX program = xelatex
\documentclass[letterpaper,12pt]{exam}
\usepackage{csc274videonotes}

\newcommand{\unit}{Unit 10}
\pagestyle{headandfoot}
\firstpageheader{CSC 274 \semester\ \  \unit}{}{Name: $\rule{6cm}{0.15mm}$}
\runningheader{CSC 274 \semester}{\unit}{Page \thepage\ of \numpages}
\firstpagefooter{}{}{}
\runningfooter{}{}{}

\begin{document}
\section*{X Windows} % Unnumbered section
If you can make X-Windows work on your system, then I suggest you have your system set up to allow X-Windows.  Windows users may need to run something like Xming or CYGWIN in order to get a server.
If you are doing ssh from the command line then you will probably need to start ssh with \code{ssh -X} so that you can run X. If you are on Putty or Bitvise you will need to enable X-11 tunneling through ssh.

\begin {questions}

\section*{10.005 Hosts} % Unnumbered section

\begin{samepage}
\question What is the purpose of the \code{hosts} file?
\vspace{5mm}
\end{samepage}

\begin{samepage}
\question What is the path to the hosts file on the computer system you are using? 
\vspace{5mm}
\end{samepage}

\section*{\unit\_010.005 -- sshKeyExchange}
\noindent For some reason the screen was not capturing properly.  

\noindent The critical things are to recognize that ssh generates a private key and a public key.  We generate a key pair and upload the .pub file to the remote system.  I will demonstrate this in class, or at least attempt to.

\section*{\unit\_010.01 -- Bash Variables}

\begin{samepage}
\question Write the code needed to set up a Bash variable called \code{NUMBER} and then echo print it to the console.
\vspace{15mm}
\end{samepage}

\begin{samepage}
\question What does the \code{env} command do? 
\vspace{5mm}
\end{samepage}

\noindent I used \textit{Command Substitution} in the statement where I set \$ENV\_SIZE.   You might want to go back and review that material if you do not remember it.

\begin{samepage}
\question What is stored in \$PS1? 
\vspace{5mm}
\end{samepage}

\begin{samepage}
\question What command would save the current contents of the \$USER variable if you wanted to mess around with it? (In the video I saved the old value of \$PS1.) 
\vspace{5mm}
\end{samepage}

\section*{\unit\_010.010 -- Part B Variables and Scripts}

\begin{samepage}
\question What command would make a file called \code{count} executable? 
\vspace{5mm}
\end{samepage}

\begin{samepage}
\question Why do methods need to export variables they create or change?  It is fair to say that unix parents are selfish with respect to their children? 
\vspace{25mm}
\end{samepage}

\section*{\unit\_010.015 -- Promptly}

\begin{samepage}
\question Write the prompt needed to create a prompt that includes the current time in 12-hour format followed by the current working directory and the dollar sign (\$). 
\vspace{5mm}
\end{samepage}

\begin{samepage}
\question What is \code{.bashrc}? 
\vspace{15mm}
\end{samepage}

\begin{samepage}
\question How do you set a custom prompt so that it is preserved between logins? 
\end{samepage}

\section*{\unit\_010.020 -- Jennifer Gardener's favorite part of Unix}

\begin{samepage}
\question What is one alias in effect when you are on woz? 
\vspace{5mm}
\end{samepage}


\noindent Again, it is not showing all the screen action.  Sorry about that.  But I think you can probably simulate what I am doing.


\begin{samepage}
\question Why is it a good idea to use the \code{which} command before you create an alias? 

\begin{samepage}
\question What is the difference between .profile and .bashrc? 
\vspace{5mm}
\end{samepage}
\end{samepage}

\begin{samepage}
\question What command gets rid of an alias? 
\vspace{5mm}
\end{samepage}

\end{questions}

\end{document}